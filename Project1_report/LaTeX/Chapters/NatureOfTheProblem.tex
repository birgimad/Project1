\section{Nature of the problem}
\label{sec:NatureOfTheProblem}
%Give a short description of the nature of the problem and the eventual numerical methods, you have used.
%"Non-computational" algebra
%Show that you can rewrite this equation as a linear set of equations of the form
The problem is to solve the one-dimensional Poisson equation with Dirichlet boundary conditions, given as
\begin{align}
	-u''(x) = p(x), \hspace{0.5cm} x\in(0,1), \hspace{0.5cm} u(0) = u(1) = 0
	\label{eq:Nature1}
\end{align}
To solve \eqref{eq:Nature1} numerically, the second derivative of $u$ is approximated by the three point formula, which gives the following reformulation of the problem.
\begin{align}
	-\frac{v_{i+1}+v_{i-1}-2v_i}{h^2} = p_i  \hspace{0.5cm} \mathrm{for} \hspace{0.1cm} i=1,\dots, n
	\label{eq:Nature2}
\end{align}
This equation can be rewritten as a linear set of equations of the form
\begin{align}
	\v{A}\v{v} = \v{f}
	\label{eq:Nature3}
\end{align}
with A beeing a $n \times n$ tridiagonal matrix given by
\begin{align}
	\v{A} = \left(\begin{array}{cccccc}
                           2& -1& 0 &\dots   & \dots &0 \\
                           -1 & 2 & -1 &0 &\dots &\dots \\
                           0&-1 &2 & -1 & 0 & \dots \\
                           & \dots   & \dots &\dots   &\dots & 									\dots \\
                           0&\dots   &  &-1 &2& -1 \\
                           0&\dots    &  & 0  &-1 & 2 \\
                      \end{array} \right)
	\label{eq:Nature4}
\end{align}
To prove this first consider the matrix equation
\begin{align}
	\v{A} = \left(\begin{array}{cccccc}
                           b_1& c_1 & 0 &\dots   & \dots & 0 \\
                           a_2 & b_2 & c_2 &\dots &\dots &\dots \\
                           0 & a_3 & b_3 & c_3 & \dots & \dots \\
                           0 & \dots   & \dots &\dots   &\dots & \dots \\
                           &   &  &a_{n-2}  &b_{n-1}& c_{n-1} \\
                           0 &  \dots  & \dots &  \dots &a_n & b_n \\
                      \end{array} \right)\left(\begin{array}{c}
                           v_1\\
                           v_2\\
                           \dots \\
                          \dots  \\
                          \dots \\
                           v_n\\
                      \end{array} \right)
  =\left(\begin{array}{c}
                           f_1\\
                           f_2\\
                           \dots \\
                           \dots \\
                          \dots \\
                           f_n\\
                      \end{array} \right).
	\label{eq:Nature5}
\end{align}
Solving the above matrix equations gives a set of linear equations as 
\begin{align*}
	2v_1 - v_2 &= f_1 \\
	-v_1 + 2v_2 -v_3 &= f_2 \\
	-v_2 + 2v_3 -v_4 &= f_3 \\
	\vdots \\
	-v_{n-1} + 2v_n -v_{n+1} &= f_n 
	%\label{eq:Nature6}
\end{align*}
In general we can write it as a 
\begin{align}
	-v_{i-1} + 2v_i -v_{i+1} &= f_i \qquad \text{for } i = 1,\dots , n
	\label{eq:Nature7}
\end{align}
If we substitute $f_i = p_i h^2$ then \eqref{eq:Nature7} becomes
\begin{align}
	-\frac{-v_{i+1}+2v_i-v_{i-1}}{h^2} = p_i  \qquad \text{for } i=1,\dots, n
	\label{eq:Nature8}
\end{align}
\eqref{eq:Nature8} is equal to \eqref{eq:Nature2} for second derivative of $u$. 
Thus proving that the equation for second derivative of $u$ can be rewritten as a set of linear equations of the form
\begin{align}
	\v{A} \v{v} = \v{f}
	\label{eq:Nature9}
\end{align}
