\section{Nature of the problem}
\label{sec:NatureOfTheProblem}
%Give a short description of the nature of the problem and the eventual numerical methods, you have used.
%"Non-computational" algebra
%Show that you can rewrite this equation as a linear set of equations of the form
We have the one-dimensional Poisson equation with Dirichlet boundary conditions as
\begin{align}
	-u''(x) = f(x), \hspace{0.5cm} x\in(0,1), \hspace{0.5cm} u(0) = u(1) = 0
	\label{eq:Nature1}
\end{align}
We approximate the second derivative of $u$ with
\begin{align}
	-\frac{v_{i+1}+v_{i-1}-2v_i}{h^2} = f_i  \hspace{0.5cm} \mathrm{for} \hspace{0.1cm} i=1,\dots, n
	\label{eq:Nature2}
\end{align}
This equation can be rewritten as a linear set of equations of the form
\begin{align}
	\v{A}\v{v} = \tilde{\v{b}}
	\label{eq:Nature3}
\end{align}
Where A is an n x n tridiagonal matrix of the form
\begin{align}
	\v{A} = \left(\begin{array}{cccccc}
                           2& -1& 0 &\dots   & \dots &0 \\
                           -1 & 2 & -1 &0 &\dots &\dots \\
                           0&-1 &2 & -1 & 0 & \dots \\
                           & \dots   & \dots &\dots   &\dots & 									\dots \\
                           0&\dots   &  &-1 &2& -1 \\
                           0&\dots    &  & 0  &-1 & 2 \\
                      \end{array} \right)
	\label{eq:Nature4}
\end{align}
To prove this first consider the matrix equation
\begin{align}
	\v{A} = \left(\begin{array}{cccccc}
                           b_1& c_1 & 0 &\dots   & \dots &\dots \\
                           a_2 & b_2 & c_2 &\dots &\dots &\dots \\
                           & a_3 & b_3 & c_3 & \dots & \dots \\
                           & \dots   & \dots &\dots   &\dots & \dots \\
                           &   &  &a_{n-2}  &b_{n-1}& c_{n-1} \\
                           &    &  &   &a_n & b_n \\
                      \end{array} \right)\left(\begin{array}{c}
                           v_1\\
                           v_2\\
                           \dots \\
                          \dots  \\
                          \dots \\
                           v_n\\
                      \end{array} \right)
  =\left(\begin{array}{c}
                           \tilde{b}_1\\
                           \tilde{b}_2\\
                           \dots \\
                           \dots \\
                          \dots \\
                           \tilde{b}_n\\
                      \end{array} \right).
	\label{eq:Nature5}
\end{align}
Solving the above matrix equations gives a set of linear equations as 
\begin{align}
	2v_1 - v_2 &= \tilde{b}_1 \\
	-v_1 + 2v_2 -v_3 &= \tilde{b}_2 \\
	-v_2 + 2v_3 -v_4 &= \tilde{b}_3 \\
	\vdots \\
	-v_{n-1} + 2v_n -v_{n+1} &= \tilde{b}_n 
	\label{eq:Nature6}
\end{align}
In general we can write it as a 
\begin{align}
	-v_{i-1} + 2v_i -v_{i+1} &= \tilde{b}_i \qquad \text{for } i = 1,\dots , n
	\label{eq:Nature7}
\end{align}
If we substitute $\tilde{b}_i = f_i h^2$ then \eqref{eq:Nature7} becomes
\begin{align}
	-\frac{-v_{i+1}+2v_i-v_{i-1}}{h^2} = f_i  \qquad \text{for } i=1,\dots, n
	\label{eq:Nature8}
\end{align}
\eqref{eq:Nature8} is equal to \eqref{eq:Nature2} for second derivative of u. 
Thus its proved that the equation for second derivative of u can be rewritten as a set of linear equations of the form
\begin{align}
	\v{A} \v{v} = \tilde{\v{b}}
	\label{eq:Nature9}
\end{align}
