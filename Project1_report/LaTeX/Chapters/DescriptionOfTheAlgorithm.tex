\section{Description of the Algorithm}
\label{sec:DescriptionOfTheAlgorithm}
%Describe the algorithm you have used and/or developed. Here you may find it convenient to use pseudocoding. In many cases you can describe the algorithm in the program itself.
%Compute number of flops
The algorithm written to solve the problem of computing vector $\v{v}$ in \eqref{eq:Nature8}, written out as the matrix equation
\begin{align*}
	\v{A} = \left(\begin{array}{cccccc}
                           b_1& c_1 & 0 &\dots   & \dots &\dots \\
                           a_2 & b_2 & c_2 &\dots &\dots &\dots \\
                           & a_3 & b_3 & c_3 & \dots & \dots \\
                           & \dots   & \dots &\dots   &\dots & \dots \\
                           &   &  &a_{n-2}  &b_{n-1}& c_{n-1} \\
                           &    &  &   &a_n & b_n \\
                      \end{array} \right)\left(\begin{array}{c}
                           v_1\\
                           v_2\\
                           \dots \\
                          \dots  \\
                          \dots \\
                           v_n\\
                      \end{array} \right)
  =\left(\begin{array}{c}
                           \tilde{b}_1\\
                           \tilde{b}_2\\
                           \dots \\
                           \dots \\
                          \dots \\
                           \tilde{b}_n\\
                      \end{array} \right).
\end{align*}
uses the Gaussian elimination method. However, since the linear problem includes a special matrix, namely a tridiagonal matrix, the number of floating points operation needed to solve this specific problem can be reduced by modifying the Gaussian elimination method.

The algorithm used to solve this problem, still, consists of the same two steps as in the Gaussian elimination: forward substitution and back substitution. 

Let us first address the forward substitution. 
The aim of the forward substitution is essentially to make the matrix $\v{A}$ into an upper triangular matrix by suitable subtractions of multiple of the first row from the other rows in the matrix. 
This gives rise to a change in the matrix elements. 
However, the elements in the first row of the matrix will not be changed, and hence we have that
\begin{align}
	\tilde{b}_1 = b_1
\end{align} 
in which $\tilde{b}_1$ is the first element in the first row of the computed triangular matrix.
By writing out the computed matrix elements after subtracting multiple of the first row from the other rows to create zeros below the diagonal, it is seen that the elements in the diagonal for $i>1$, named $b_i$ in the tridagonal matrix $\v{A}$ and $\tilde{b}_i$ in the computed triangular matrix, get the value
\begin{align}
	\tilde{b}_i = b_i - \frac{1}{b_{i-1}}
	\hspace{0.5cm} \mathrm{for} \hspace{0.1cm} i=2,\dots, n
	\label{eq:Alg1}
\end{align}
\eqref{eq:Alg1} has that form since all elements right above and below the diagonal of $\v{A}$ are equal to $-1$, the ones in the diagonal of $\v{A}$ are $b_i$, whilst the remaining elements of the matrix are $0$.
Likewise, the elements in the vector $\v{f}$ are changed to
\begin{align}
	\tilde{f}_i = f_i + \frac{f_{i-1}}{b_{i-1}}
	\hspace{0.5cm} \mathrm{for} \hspace{0.1cm} i=2,\dots, n
	\label{eq:Alg2}
\end{align}
whilst the elements named $a_i$ in $\v{A}$ become equal to zero, and the elements $c_i$ are unchanged. 

This gives rise to the following code for the forward substitution.

\begin{lstlisting}
// Forward substitution

    double abtemp[n];
    double btemp = b[0];

    for (int i=1 ; i<n ; i++)
    {
        abtemp[i] = - 1/btemp;
        btemp = b[i] + abtemp[i];
        f[i] = f[i] - f[i-1]*abtemp[i];
        b[i] = btemp;
    }
\end{lstlisting}
%\fxnote{skriv evt. -= osv i stedet}
Notice that in the above lines of code, the first element of a vector is $i=0$.

For every time the for loop runs, there are 4 \flops\ . We have chosen to calculate $1/b_{i-1}$, which is used in both \eqref{eq:Alg1} and \eqref{eq:Alg2}, as the very first operation in the for loop to reduce the number of \flops \, by 1 for every time the loop is run. 
Since the loop runs for $i=2$ to $i=n$, if $i=1$ is the first element of a vector (remember, the c++ code above has $i=0$ for $i=1$), the loop runs $n-1$ times, which gives a total number of \flops \, for the forward substitution of
\begin{align}
	 \# flops = 4(n-1)
	 \label{eq:Alg3}
\end{align} 
In the back substitution, the values of the entrances of vector $\v{v}$ in \eqref{eq:Nature9} are computed. 
Since the result from the forward substitution is an upper triangular matrix, it is evident that
\begin{align}
	v_n = \frac{\tilde{f}_n}{\tilde{b}_n}  
	\label{eq:Alg4}
\end{align}
in which $\tilde{f}_n$ and $\tilde{b}_n$ are elements of $\v{f}$ and $\v{b}$ after the forward substitution. 
From the determined value of $v_n$, the values of the rest of the $v_i$'s can be determined using the fact, that all elements in the upper triangular matrix are zero apart from the elements in the diagonal and the elements just above the diagonal. 
Since the elements just above the diagonal are still the $c_i$'s, they have the value $-1$, which yields that the value of $v_i$ can be determined by 
\begin{align}
	v_i = \frac{\tilde{f}_i + v_{i+1}}{\tilde{b}_i} 
	\hspace{0.5cm} \mathrm{for} \hspace{0.1cm} i=1,\dots, n-1
	\label{eq:Alg5}
\end{align}
yielding a source code:
\begin{lstlisting}
// Back substitution

    v[n-1] = f[n-1]/b[n-1];

    for(int i=n-1 ; i>= 0; i--)
    {
        v[i] = (f[i]+v[i+1])/b[i];
    }
\end{lstlisting} 
Once again, the first element of a vector in the source code above has the index $i=0$. Hence, $v[n-1]$ is the same as $v_n$.
\\
For each time the loop runs, there are $2$ \flops\ . Like in the forward substitution, the loop runs $n-1$ times, yielding
\begin{align}
	 \# flops = 2(n-1)+1
	 \label{eq:Alg6}
\end{align} 
The number $+1$ is included in \eqref{eq:Alg6} due to the initial calculation of $v_n$ just before the for loop.
Hence, the total number of \flops \, for both the forward substitution and back substitution is
\begin{align}
	 \# flops_{total} = 4(n-1)+2(n-1)+1 = 6n-6 
	 \label{eq:Alg7}
\end{align}  
which gives that the number of \flops \, goes as $6n$ or $\mathcal{O}(n)$.

When comparing the number of \flops \, using the above described algorithm to solve \eqref{eq:Nature8} in which $\v{A}$ is a tridiagonal matrix and $\v{v}$ is approximated by the three point formula \eqref{eq:Nature2} with the number of \flops \, in the ordinary Gaussian elimination or LU decomposition, it is evident that this algorithm is much more efficient to solving this specific case, since the number of \flops \, for solving a linear set of equations using the LU decomposition scales as $\mathcal{O}(n^2)$, whilst the LU decomposition itself scales as $\mathcal{O}(n^3)$, and the number of \flops \, required in the Gaussian elimination is $2n^3 /3 +\mathcal{O}(n^2)$. \cite[173]{CompLectureNotes}



