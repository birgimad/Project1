\chapter{Introduction}
%An introduction where you explain the aims and rationale for the physics case and what you have done. At the end of the introduction you should give a brief summary of the structure of the report

The problem of solving differential equations appear in various places in nature. 
It is, however, not always possible to solve these differential equations analytically, and one might instead need solve the problem numerically. 

This project aims to solve the problem of finding a solution for the general one-dimensional Poisson equation with the double differentiated function approximated by the three point formula using less floating point operations (\flops\ ) than needed in the ordinary Gaussian elimination and the LU decomposition method. 
The number of \flops \, is reduced by identifying that the Poisson equation can be numerically solved by solving a matrix equation in which the matrix is tridiagonal. 
This ultimately ends out in a number of \flops \, that goes as $\mathcal{O}(n)$ for this specific problem which is evidently a great reduction from the number of \flops \, needed in the other two methods.

Futhermore, another objective of the project is to consider how a decrease of step length of the variable $x$ influences the precision of the numerical solution compared to the closed-form solution. 
It is seen that for the step lengths chosen, a decrease in step length causes an increase in accuracy of the solution.   

